\documentclass[a4paper]{article}
\usepackage[14pt]{extsizes} % для того чтобы задать нестандартный 14-ый размер шрифта
\usepackage[utf8]{inputenc}
\usepackage[russian]{babel}
\usepackage[left=2cm, right=2cm,
top=1.5cm, bottom=2cm]{geometry}% Change the margins here if you wish.
\setlength{\parindent}{0pt} % This is the set the indent length for new paragraphs, change if you want.
\setlength{\parskip}{5pt} % This sets the distance between paragraphs, which will be used anytime you have a blank line in your LaTeX code.
\usepackage[russian]{babel}
\usepackage{amsmath}
\usepackage{amsfonts}
\usepackage{amssymb}
\usepackage{listings}
\usepackage{amsfonts}
\usepackage{graphicx}
\usepackage{setspace}
\usepackage{ragged2e}
\usepackage{blindtext}
\usepackage[normalem]{ulem}
\justifying

% полуторный интервал
\onehalfspacing
\graphicspath{{}}
\setlength{\parskip}{1.1em}


\title{Пояснительная записка к \\Домашнему заданию №1 по курсу \\Архитектура вычислительных систем}
\date{Сентябрь-Октябрь\\2021}
\author{Шагаров Дмитрий Александрович
\\БПИ202}

\begin{document}
\maketitle
\newpage

\begin{center}
\section*{Описание полученного задания}
\end{center}
\begin{center}
    \begin{tabular}{ |c|c|c| }
        \hline
       номер варианта & номер задания & номер функции \\
        \hline
       70 & 14 & 5 \\
        \hline
    \end{tabular}
\end{center}
\subsection*{Разработка сущностей}
\begin{enumerate} 
    \item \textbf{Грузовик} (грузоподъемность кг – целое, емкость
    топливного
    бака в литрах
    (целое), расход
    топлива на
    100 км в литрах (действительное))
      
    \item \textbf{Автобус} (пассажировместимость – короткое целое, емкость
    топливного
    бака в литрах
    (целое), расход
    топлива на
    100 км в литрах (действительное))

    \item \textbf{Легковой автомобиль} (максимальная скорость –
    короткое целое, емкость
    топливного
    бака в литрах
    (целое), расход
    топлива на
    100 км в литрах (действительное))
\end{enumerate}
Каждая сущность имеет функцию находения максимального
расстояния,
которое может
пройти автомобиль в км
(действительное число).

Все введеные сущности размещаются в разработанном контейнере, после чего к контейнеру применяется сортировка Шелла (ключ - значение функции максимального расстояния). Элементы контейнера до и после сортировки выводется в форматируемый поток.

\newpage
\subsection*{Формат ввода}
В программе предусмотрено два способа ввода данных при запуске из командной строки:
\begin{enumerate} 
    \item \textbf{Использование генераторов случайных наборов данных} - \\команда \\\verb|./AoCS -n number_of_vehicles out_path sorted_out_path|
    
    \item \textbf{Ввод из заранее подготовленных тестовых
    файлов} - \\команда \\\verb|./AoCS -f in_path out_path sorted_out_path|
\end{enumerate}

Второй способ предусматривает наличие файла in с описанием сущностей в формате:
"одна строка = одно ТС.\\
Первый параметр - число от 1 до 3, где 1 - легковое авто, 2 - автобус, 3 - грузовик.\\
Второй параметр - уникальное для каждого ТС свойство
80 - 320 целое для легкового авто, 10 - 70 целое для автобуса, 1000 - 4000 целое для грузовика.\\
Третий параметр - объем топливного бака
50 - 150 целое для легкового авто, 80 - 250 целое для автобуса, 200 - 800 целое для грузовика.\\
Последний параметр - расход
6 - 25 вещественное для легкового авто, 12 - 30 вещественное для автобуса, 20 - 45 вещественное для грузовика.\\
Параметры указываются через один пробел, для вещественных разделитель - точка. Окончание файла - пустая строка.\\
Пример такого файла - \verb|in_example.txt| находится в папке с проектом. Для корректной работы программы необходимо наличие файла, передаваемого как \verb|in_path|.
\newpage
\begin{center}

\section*{Структурная схема программного продукта\\с использованием процедурного подхода\\ и статической типизацией}
\end{center}

\subsection*{Таблица типов}

\begin{tabular}{ |c|c| }
    \hline
    short, int, double & 2 байта, 4 байта, 8 байт \\
    \hline
    \underline{struct Car} & \underline{14 байт} \\
    int tank\verb|_|volume & 4 байта[0] \\
    short max\verb|_|speed & 2 байта[4]\\
    double consumption & 8 байт[6]\\
    \hline
    \underline{struct Bus} & \underline{14 байт} \\
    int tank\verb|_|volume & 4 байта[0] \\
    short max\verb|_|passengers & 2 байта[4]\\
    double consumption & 8 байт[6]\\
    \hline
    \underline{struct Truck} & \underline{16 байт} \\
    int max\verb|_|weight, tank\verb|_|volume & 8 байт[0, 4] \\
    double consumption & 8 байт[8]\\
    \hline
    \underline{struct Vehicle} & \underline{52 байта} \\
    enum key & 4 байта[0] \\
    key k & 4 байта[4] \\
    Car car & 14 байт[8]\\
    Bus bus & 14 байт[22]\\
    Truck truck & 16 байт[36]\\
    \hline
    \underline{struct Container} & \underline{52 байта} \\
    enum max\verb|_|len & 4 байта[0] \\
    int len & 4 байта[4] \\
    Vehicle cont & 520052 байта[8] (= 52 * 10001)\\
    \hline
\end{tabular}

\subsection*{Память программы}
\begin{tabular}{ |c|c| }
    \hline
    main(int argc, char *argv[]) &  \\
    int argc & 4 байта[0]\\
    char *argv & 8 байт[4]\\
    Container c &  520052 байта[8]\\
    int size & 4 байта[520064]\\
    \hline
    void StartMessage & \\
    \hline
    void ErrMessage1 & \\
    \hline
    void ErrMessage2 & \\
    \hline
    void Init(Container \verb|&|cont) & \\
    \hline
    void In(Container \verb|&|cont, FILE *input) & \\
    \hline
    void InRnd(Container \verb|&|cont, int size) & \\
    \hline
    void ShellSortByMaxDistance(Container \verb|&|c) & \\
    int d, i, j & 12 байт[0, 4, 8]\\
    Vehicle *temp & 52 байта[12]\\
    \hline
    double MaxDistance(Vehicle \verb|&|vehicle) & \\
    \hline
    void Out(Container \verb|&|cont, FILE *output) & \\
    int i & 4 байта[0]\\
    \hline
    void Clear(Container \verb|&|cont) & \\
    int i & 4 байта[0]\\
    \hline
\end{tabular}

\newpage
\subsection*{Стек вызовов}
Возможны следуюшие варинты ( | означает что вызыается одна из соответствующих функций, ? ? $\rightarrow$ - что будет, если программа вызовет данную функцию)
\\
\begin{tabular}{ ||c|| }
    \hline
    main\\
    \hline
    ?ErrMessage1? $\rightarrow$ \sout{main}\\
    \hline
    StartMessage\\
    \hline
    \sout{StartMessage}\\
    \hline
    Init\\
    \hline
    \sout{Init}\\
    \hline
    In (error $\rightarrow$ \sout{main}) | InRnd | ?ErrMessage2? $\rightarrow$ \sout{main}\\
    \hline
    \sout{In} | \sout{InRnd}\\
    \hline
    Out\\
    \hline
    \sout{Out}\\
    \hline
    ShellSortByMaxDistance\\
    \hline
    \sout{ShellSortByMaxDistance}\\
    \hline
    Out\\
    \hline
    \sout{Out}\\
    \hline
    Clear\\
    \hline
    \sout{Clear}\\
    \hline
    \sout{main}\\
    \hline
\end{tabular}

\newpage
\section*{Основные характеристики программы}

\begin{tabular}{ |c|c| }   
    \hline   
    интерфейсных модулей & модулей реализации\\
    \hline
    6 & 6 ( + main.cpp для тестирования)\\    
    \hline
\end{tabular}


\begin{tabular}{ |c| }   
    \hline   
    общий размер исходных текстов\\
    \hline
    17,37 Кб\\
    \hline
\end{tabular}

\end{document}